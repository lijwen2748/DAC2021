\section{Concluding Remarks}
In this paper, we apply the restart policy to Hardware Model Checking, aiming to get rid of the trap, which occurs during search and make the algorithm not terminate in a reasonable time. The results of the experiments show that the restart policy increases the diversity of the CAR algorithm, though there is no single configuration that can improve the overall performance significantly. The new finding of 13 unsafe instances indicates the efficiency of the restart policy in the domain. Moreover, the CAR algorithm with the restart policy can now find 18 unsafe instances with counterexamples that the BMC can not find, which enhances the ability of the current hardware model checking portfolio. 

This is the first work to understand how the restart policy performs on hardware model checking.
In future work, we plan to design more elaborate and sophisticated restart mechanisms to improve the overall performance of CAR such that it is able to outperform BMC on bug-finding with a single restart configuration. Due to the fact that different problems are sensitive to different restart frequencies, it is also interesting to introduce the learning techniques to learn a best solution for different instances. 


\iffalse
\begin{thebibliography}{00}
\bibitem{b1} Biere, A., Cimatti, A., Clarke, E., Zhu, Y.: Symbolic model checking without BDDs. In: Cleaveland, W.R. (ed.) Tools and Algorithms for the Construction and Analysis of Systems. pp. 193–207. Springer Berlin Heidelberg, Berlin, Heidelberg (1999)

\bibitem{b2} Li, J., Zhu, S., Zhang, Y., Pu, G., Vardi, M.Y.: Safety model checking with complementary approximations. In: Proceedings of the 36th International Conference on Computer-Aided Design, 2017, pp. 95–100

\bibitem{b3} Li, Jianwen, et al. "Simplecar: An efficient bug-finding tool based on approximate reachability." International Conference on Computer Aided Verification. Springer, Cham, 2018.

\bibitem{b4} Huang, Jinbo. "The Effect of Restarts on the Efficiency of Clause Learning." IJCAI. Vol. 7. 2007.

\bibitem{b5} Audemard, Gilles, and Laurent Simon. "Refining restarts strategies for SAT and UNSAT." International Conference on Principles and Practice of Constraint Programming. Springer, Berlin, Heidelberg, 2012.

\bibitem{b6} Een, N., Mishchenko, A., Brayton, R.: Efficient implementation of property directed reachability. In: Proceedings of the International Conference on Formal Methods in ComputerAided Design. pp. 125–134. FMCAD ’11, FMCAD Inc, Austin, TX (2011), http://dl. acm.org/citation.cfm?id=2157654.2157675

\bibitem{b7} Bradley, A.R.: Sat-based model checking without unrolling. In: Jhala, R., Schmidt, D. (eds.) Verification, Model Checking, and Abstract Interpretation. pp. 70–87. Springer Berlin Heidelberg, Berlin, Heidelberg (2011)

\bibitem{b8} McMillan, K.L.: Interpolation and sat-based model checking. In: Hunt, W.A., Somenzi, F. (eds.) Computer Aided Verification. pp. 1–13. Springer Berlin Heidelberg, Berlin, Heidelberg (2003)

\bibitem{b9} Biere, A., Cimatti, A., Clarke, E.M., Fujita, M., Zhu, Y.: Symbolic model checking using sat procedures instead of bdds (1999), http://doi.acm.org/10.1145/309847.309942

\bibitem{b10} HWMCC 2015. http://fmv.jku.at/hwmcc15/

\bibitem{b11} HWMCC 2017. http://fmv.jku.at/hwmcc17/

\bibitem{b12} Brayton, Robert, and Alan Mishchenko. "ABC: An academic industrial-strength verification tool." International Conference on Computer Aided Verification. Springer, Berlin, Heidelberg, 2010.
\end{thebibliography}
\fi
